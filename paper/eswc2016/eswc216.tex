%%%%%%%%%%%%%%%%%%%%%%%%%%%%%%%%%%%%%%%%%%%%%%%%%%%%%%%%%%%%%%%%%%%%%%%%%%%%%%%%%%%%%%%%µ%%%%%%%%%%%%%%%%%%
%%%  Automatic Domain Identification of Vocabularies: Issues and Challenges %%%%
%%%%%%%%%%%%%%%%%%%%%%%%%%%%%%%%%%%%%%%%%%%%%%%%%%%%%%%%%%%%%%%%%%%%%%%%%%%%%%%%%%%%%%%%%%%%%%%%%%%%%%%%%%%

\documentclass[runningheads,a4paper]{llncs}

\usepackage[utf8]{inputenc}
\usepackage{amssymb}
\setcounter{tocdepth}{3}
\usepackage{graphicx}
\usepackage{tabularx}
\usepackage{url}
\usepackage{listings}
\usepackage{subfigure}
\usepackage{algorithmic}
\usepackage{algorithm}

\newcommand{\keywords}[1]{\par\addvspace\baselineskip
\noindent\keywordname\enspace\ignorespaces#1}

% todo macro
\usepackage{color}
\newtheorem{deflda}{Axiom}
\newcommand{\todo}[1]{\noindent\textcolor{red}{{\bf \{TODO}: #1{\bf \}}}}

% Language Definitions for Turtle
\definecolor{olivegreen}{rgb}{0.2,0.8,0.5}
\definecolor{grey}{rgb}{0.5,0.5,0.5}
\lstdefinelanguage{ttl}{
sensitive=true,
morecomment=[l][\color{brown}]{@},
morecomment=[l][\color{red}]{\#},
morestring=[b][\color{blue}]\",
}

%%%%%%%%%%%%%%%%%%%%%%%%%%%%%%%
%%%  Beginning of document  %%%
%%%%%%%%%%%%%%%%%%%%%%%%%%%%%%%

\begin{document}


%\title{Semantic Technologies for Matching Operational Interruptions to Tasks in Aircraft Industry }
\title{Semantic Technologies for Revealing Relevant Operational Interruptions in
Aircraft Industry}
%\subtitle{Classifying vocabularies }

\author{ Ghislain Auguste Atemezing\inst{1}, Dang Nguyen-manh\inst{2} }

\institute{
MONDECA, 35 Boulevard de Strasbourg, Paris, France. \\
\and Airbus, Toulouse, France.\\
\email{ghislain.atemezing@mondeca.com} \\
\email{dang.nguyeng-manh@airbus.com} \\
}
  


% a short form should be given in case it is too long for the running head
\titlerunning{Semantics Applied to Detecting OIs in Aircraft Industry}	


\maketitle

%%%%%%%%%%%%%%%%%%
%%%  Abstract  %%%
%%%%%%%%%%%%%%%%%%

\begin{abstract}

Airbus, one of the leading Aircraft company in Europe, collects and manages  a substantial amount of unstructured data from airlines companies, related to events occurring during the exploitation of an aircraft. Those events are called ``Operational Interruptions'' (OI) describing observations, such as ``noises in the radio'' and the work performed associated by an operator, like ``I have replaced the component''. At the same time, Airbus maintains a dataset of programmed maintenance task (MPD) for each family of aircraft.  Currently, the information of the OIs is gathered in Excel spreadsheets and experts have to find manually in the OIs the ones that are most likely to match an existing task. In this paper, we describe a semi-automatic approach using semantic technologies to assist the experts of the domain to improve the matching process of OIs with related MPD. Our approach combines text annotation using GATE based on an RDF model and a graph matching algorithm using SPARQL, thanks to the datasets converted in RDF. The evaluation of the approach shows the benefits of using semantic technologies to manage unstructured data at Airbus.    

\keywords{Information Retrieval, Graph matching, CA-Manager, GATE, Airbus}
\end{abstract}

%%%%%%%%%%%%%%%%%%%%%%%%%
%%%  1. Introduction  %%%
%%%%%%%%%%%%%%%%%%%%%%%%%



\section{Introduction}
\label{sec:introduction}

\todo{add the context of the project}

%%%%%%%%%%%%%%%%%%%%%%%%%%%%%%%%%%%%%%%%%%%%%%%%%%%%%%%%%%%%%%%%%
%%%  2. Manual curation of domain in LOV  %%%
%%%%%%%%%%%%%%%%%%%%%%%%%%%%%%%%%%%%%%%%%%%%%%%%%%%%%%%%%%%%%%%%%

%\section{Domain Classification Approaches}
%\label{sec:background}

Show the problem and implication

\section{Related Work}
\label{sec:soa}

Work of domain classification in general, with vocabularies as well \\

\section{Manual Curation of Domain vocabularies}
\label{sec:curation}

Take a vocab, look at the content and classes..manual identification of the domain.. assignment and/or creation of the domain in LOV backend. Applications: visual in the front-end and in the results page. 

\begin{algorithm}[h]\scriptsize
\caption{Process for detecting categories for vocabularies} \label{experiment}
\begin{algorithmic}[1]
    \STATE INITIALIZE $equivalentClasses(DBpedia,Freebase) $ AS $vectorClasses$
    \STATE Upload $vectorClasses$ for querying processing
    \STATE Set $n$ AS number-of-instances-to-query
    \FOR {each $conceptType \in vectorClasses$}
	\STATE SELECT $n$ instances
	\STATE $listInstances \leftarrow$ SELECT-SPARQL($conceptType$, $n$)
		\FOR {each $instance \in listInstances$}
			\STATE CALL http://www.google.com/search?q=$instance$
			\IF {$knowledgePanel$ exists}
				\STATE SCRAP GOOGLE KNOWLEDGE PANEL
			\ELSE
				\STATE CALL http://www.google.com/search?q=$instance + conceptType$
 				\STATE SCRAP GOOGLE KNOWLEDGE PANEL
			\ENDIF
			\STATE $gkpProperties \leftarrow$ GetData(DOM, EXIST(GKP))
			
		\ENDFOR
	\STATE COMPUTE occurrences for each $prop \in gkpProperties$
    \ENDFOR
    \RETURN $gkpProperties$
\end{algorithmic}
\end{algorithm}


%%%%%%%%%%%%%%%%%%%%%%%%%%%%%%%%%%%%%%%%%%%%%%%%%
%%%  3. Domain classification using Alchemy  %%%
%%%%%%%%%%%%%%%%%%%%%%%%%%%%%%%%%%%%%%%%%%%%%%%%%

\section{Experiments and Evaluation}
\label{sec:classification}
Explain the experiments..describe the Alchemy output and findings

Query in LOV aggregator for vocabularies + description + domain (inLOV) at \url{http://goo.gl/uJgQ2w}\footnote{Dataset as of 6th, November 2014 with 475 vocabularies.}

\subsection{Dataset preparation}

\subsection{Classifier}

\subsection{Results}

\section{Discussion}
\label{sec:discussion}




%%%%%%%%%%%%%%%%%%%%%%%%%%%%%%%%%%%%%%%
%%%  4. Conclusion and Future Work  %%%
%%%%%%%%%%%%%%%%%%%%%%%%%%%%%%%%%%%%%%%

\section{Conclusion and Future Work}
\label{sec:conclusion}
\todo{go for a tool to categorize and tag vocabularies}
Everything is writing here the ``take-away message''


%%%%%%%%%%%%%%%%%%%%%%%%%
%%%  Acknowledgments  %%%
%%%%%%%%%%%%%%%%%%%%%%%%%

\paragraph{\textbf{Acknowledgments.}} %\label{sec:acknowledgments}
We would like to thank the Airbus staff DANG Nguyen and colleagues at ATOS Toulouse for their valuable input and comments.

\bibliographystyle{abbrv}
\nocite{*}
\bibliography{eswc2016}
\end{document}
